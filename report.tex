\documentclass{report}
\usepackage[utf8]{inputenc}
\usepackage{fancyhdr}
\usepackage{graphicx}
\usepackage{amsmath}
\usepackage{amsfonts}
\usepackage{hyperref}
\def\thesection{\arabic{section}}
\def\thetableofcontents{\arabic{section}}

\pagestyle{fancy}
\fancyhead{}
\fancyhead[L]{Johan Wikström(jwikstrom) - 645714\\
Anael Bonneton(abonneton) - }
\fancyfoot{}

\title{JavaScript and its suitability as an application development language}
\author{Johan Wikström(jwikstrom) - 645714\\
Anael Bonneton(abonneton) - 
}
\begin{document}
\maketitle
\tableofcontents
\section{Introduction}
Is JavaScript suitable for application programming?


The DOM (Document Object Model) is the official W3C standard for accessing HTML elements. 

\section{Syntax and semantics}
    
        -blocks
            function, 

var carname="Volvo";
var carname;
    -> no error, the variable carname has still the value "Volvo"

creating an object -> new
    -> in JavaScript everything is an object
    -> different ways to create an object

    ## with a constructor
function person(firstname,lastname,age,eyecolor)
{
this.firstname=firstname;
this.lastname=lastname;
this.age=age;
this.eyecolor=eyecolor;
}
var myFather=new person("John","Doe",50,"blue");
var myMother=new person("Sally","Rally",48,"green");


## Way to define a class (like in Java but everything in a constructor)
## In JavaScript you don’t define classes and create objects from these classes (as in most other object oriented languages).
## JavaScript is prototype based, not class based.
function person(firstname,lastname,age,eyecolor)
{
this.firstname=firstname;
this.lastname=lastname;
this.age=age;
this.eyecolor=eyecolor;

this.changeName=changeName;
function changeName(name)
{
this.lastname=name;
}
}

    ## person={firstname:"John",lastname:"Doe",age:50,eyecolor:"blue"}; 
    ## 
person=new Object();
person.firstname="John";
person.lastname="Doe";
person.age=50;
person.eyecolor="blue"; 

	- scope
The lifetime JavaScript variables starts when they are declared.
Local variables are deleted when the function is completed.
Global variables are deleted when you close the page.


	- semicolon
Ending statements with semicolon is optional in JavaScript.
	- this
	- object vs function
	- multi paradigm

\section{Security}
	- XSS
	- Compare with Java
	- Sandboxing
        - the importance of validation
JavaScript can be used to validate data in HTML forms before sending off the content to a server.
Form data that typically are checked by a JavaScript could be:
    has the user left required fields empty?
    has the user entered a valid e-mail address?
    has the user entered a valid date?
    has the user entered text in a numeric field?




\section{Type System}
	JavaScript has a loose dynamic type system. This means that when you create a variable, you do not need to specify a type and the type of the variable can change during execution. Internally there is a type however and there are six internal types:
	%http://www.2ality.com/2013/09/types.html
	\begin{enumerate}
	\item Undefined
	\item Null
	\item Boolean
	\item String
	\item Number
	\item Object
	\end{enumerate}
		\begin{verbatim}
	Java
	int x = 1;
	x = "hi";   // invalid, causes compiler error
	\end{verbatim}
	\begin{verbatim}
	JavaScript
	var x = 1;  // now x has Number type
	x = "hi";   // now x has String type
	x = {};     // now x has Object type
	\end{verbatim}
	If a variables type does not fit the operation that is applied on it, JavaScript will attempt to \emph{corece} the variable into that type.
	\begin{verbatim}
	'1'+'10' == 11 // true in javascript
	\end{verbatim}
	In general, JavaScript tries to recover from errors made by the programmer, by almost never throwing exceptions when encountering type errors. For example, when an uninitialized variable is used, there is no error but the variable is assigned the type Undefined and value undefined. This usually causes errors much later in the program when the value is used. The errors are exaggerated by the fact that several normal operations are defined for the undefined value which further delays detection of the error.
	\begin{verbatim}
	1+undefined == NaN
	''+undefined == 'undefined'
	undefined && true == undefined
	undefined || true == true
	\end{verbatim}
This loose typing may be a disadvantage of the language when comparing the language to strongly typed languages such as Java. In Java the type system allows for static analysis of the code and many bugs are found at compile time. The true advantage of the loose type system only becomes apparent when the programmer uses the ability to augment types with new methods and properties at runtime.

	\subsection{JavaScript Object and prototyping}
	In Java, classes are static. There is no way to add a new method or property to an object after it is created. In JavaScript, there are no classes so all methods and properties are dynamically. This is a very powerful feature that may also be the cause of many bugs since properties and methods can be added and removed at any time during the programs execution.
\begin{verbatim}
JavaScript
a = {} 	// empty object
a.property = "a"
a.method = function(){print("hello")}
\end{verbatim}

\begin{verbatim}
Java
class A {
	public String property;
	public void method(){
		print("hello");
	}
}
A a = new A();
a.property = "a";
a.property2 = "b"; // Invalid, was not specified in class
\end{verbatim}

Another difference between Java and JavaScript is the way inheritance is implemented. In Java, a class can statically inherit methods and variables from a parent class. In JavaScript, there are no classes, only dynamic objects, so JavaScript uses another form of inheritance. All objects in a JavaScript execution are linked to a prototype object. The prototype describes the methods and properties assigned to the object and this is used by the dynamic type system to determine whether 
	One feature
	- inheritance
	- loose typing
	- dynamic
\section{Application design}
	Global object
	code structure/design
\end{document}
