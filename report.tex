\documentclass{report}

\usepackage{extension}
\pagestyle{fancy}
\fancyhead{}
\fancyhead[L]{Johan Wikström, Anaël Bonneton}
\fancyfoot{}

\title{JavaScript and its suitability as an application development language}
\author{Johan Wikström (jwikstrom) - 645714\\
Anaël Bonneton (abonneton) - 646275 
}
\begin{document}
\maketitle
\tableofcontents

\section{Introduction}
\paragraph{}
\texttt{JavaScript} was created by Brendan Eich in 1995 for the company Netscape Communications Corporation. Today, it is the most used programming language for client-side programming on the web\cite{w3techs}. It has also become common in server-side programming. However, \texttt{JavaScript} has a bad reputation among professional developers. The main reason is that it was initially created to make programming easier than with traditional programming languages such as \texttt{Java}. \texttt{JavaScript} was created for beginner programmers and had several features that were not appealing to professional developers.
\paragraph{}
In this report we will answer the following question \emph{"Is JavaScript suitable for application programming?"}. We will focus on three areas: application design, the type system and the security issues.


%%
%%
%%The DOM (Document Object Model) is the official W3C standard for accessing HTML elements. 
%%
\section{Application design}

\paragraph{}
The \texttt{JavaScript} syntax is easy, without a lot of contraints like in \texttt{Java}. For example, semi-colons are not compulsary at the end of each statement.

\subsection{Semi-colons}

\subsection{Global and local variables}

\subsection{Creating an object}
\paragraph{}
\texttt{JavaScript} offers several ways to create an object with its fields and its methods. The listings~\ref{way1},~\ref{way2} and~\ref{way3} show the different options.
        
\begin{lstlisting}[caption={Creating an object with a constructor}, label={way1}]
function coordinates(x, y) {
    // Definition of the fields
    this.x = x;
    this.y = y;
    // Definition of the methods
    this.setX = setX;
    function setX(x) {
        this.x = x;
    }
    this.setY = setY;
    function setY(y) {
        this.y = y;
    }
}
var point = new coordinates(1, 2);
\end{lstlisting}
        
\begin{lstlisting}[caption=Creating an object with the "C-structure" syntax, label={way2}]
// We suppose the functions setX and setY are defined before
var point = {x:1, y:2, setX:setX, setY:setY};
\end{lstlisting}

\begin{lstlisting}[caption=Creating an object with the operator "new Object()", label={way3}]
var point = new Object();
point.x = 1;
point.y = 2;
// We suppose the functions setX and setY are defined before
point.setX = setX;
point.setY = setY;
\end{lstlisting}

Whatever the method used to create the object, it is always possible to add another field or another method dynamically (listing~\ref{addField}).
\begin{lstlisting}[caption=Adding a new field or a new method dynamically", label={addField}]
point.z = 3;
function setZ(z) {
        this.z = z;
    }
point.setZ = setZ;
\end{lstlisting}

\paragraph{}
With the "C-structure" syntax and the operator \texttt{new Object()} it is very easy to develop a \texttt{JavaScript} application without thinking about the design of the application. Non-experienced developers may write code very quickly without thinking about readability for future developers and maintainability. However, the \texttt{JavaScript} constructors can be used as the classes in \texttt{Java} to define all the fields and the methods of an object.


%%	Global object
%%	code structure/design
%%    Syntax and semantics
%%        -blocks
%%            function, 
%%
%%var carname="Volvo";
%%var carname;
%%    -> no error, the variable carname has still the value "Volvo"
%%
%%
%%	- scope
%%The lifetime JavaScript variables starts when they are declared.
%%Local variables are deleted when the function is completed.
%%Global variables are deleted when you close the page.
%%
%%
%%	- semicolon
%%Ending statements with semicolon is optional in JavaScript.
%%	- this
%%	- object vs function
%%	- multi paradigm
%%



\section{Type System}

\paragraph{}
\texttt{JavaScript} has a loose, dynamic type system. This means that when you create a variable, you do not need to specify a type and the type of the variable can change during execution as shown in listings \ref{javatypes} and \ref{javascripttypes} . Internally there is a type however and there are six built-in types\cite{2ality}:
	\begin{enumerate}
	\item Undefined
	\item Null
	\item Boolean
	\item String
	\item Number
	\item Object
	\end{enumerate}
        \begin{lstlisting}[caption=Changing the type of a variable in \texttt{Java},label={javatypes}]
int x = 1;
x = "hi";   // invalid, causes compiler error
\end{lstlisting}

\begin{lstlisting}[caption=Changing the type of a variable in \texttt{JavaScript},label={javascripttypes}]
var x = 1;  // now x has Number type
x = 'hi';   // now x has String type
x = {};     // now x has Object type
\end{lstlisting}
\paragraph{}
	If a variable's type does not fit the operation that is applied to it, \texttt{JavaScript} will attempt to \emph{coerce} the variable into that type. This can be a source of great confusion and many bugs since it effectively masks errors. However it can make programming easier for beginners since they do not have to worry about bitness, converting input strings into numbers and some sorts of zero inputs.
\begin{lstlisting}[caption=Automatic type coercion]
2 + '10'   == '210' // Number coerced into String
2 * '10'   == 20    // String coerced into Number
true - 10  == -9    // Boolean coerced into Number
\end{lstlisting}

\paragraph{}
In general, \texttt{JavaScript} tries to recover from errors made by the programmer, by avoiding throwing exceptions when encountering type errors. For example, when an uninitialized variable is used, there is no error : the variable is assigned the type \texttt{Undefined} and the value \texttt{undefined}. This usually causes errors later in the program when the value is used. The errors are amplified by the fact that several normal operations are defined for the undefined value (listing~\ref{undefined}) which further delays error detection.
\begin{lstlisting}[caption=Normal operations on undefined value,label=undefined]
1+undefined       == NaN
''+undefined      == 'undefined'
undefined && true == undefined
undefined || true == true
\end{lstlisting}	
\paragraph{}
This loose typing may be a disadvantage of the language when comparing the language to strongly typed languages such as \texttt{Java}. In \texttt{Java} the type system allows for static analysis of the code and many bugs are found at compilation time. 

%%%% TODO I do not understand
The true advantage of the loose type system only becomes apparent when the programmer uses the ability to augment types with new methods and properties at runtime.

\subsection{JavaScript Object and prototyping}
\paragraph{}
In \texttt{Java}, classes are static. There is no way to add a new method or property to an object after it has been created. In \texttt{JavaScript}, there are no classes : all fields and methods are dynamic. This is a very powerful feature that may also be the cause of many bugs since fields and methods can be added and removed at any time during the program execution. This source of errors is not present in \texttt{Java} as shown in listings \ref{dynclasses} and \ref{statclasses}.

\begin{lstlisting}[caption=JavaScript,label=dynclasses]
a = new Object(); // empty object
a.property = "a";
a.method = function(){print("hello")}
\end{lstlisting}	

\begin{lstlisting}[caption=Java,label=statclasses]
class A {
	public String property;
	public void method(){
		print("hello");
	}
}
A a = new A();
a.property = "a";
a.property2 = "b"; // Compilation error, was not specified in class
\end{lstlisting}
\paragraph{}
Another difference between \texttt{Java} and \texttt{JavaScript} is the way inheritance is implemented. In \texttt{Java}, a class can statically inherit methods and variables from a parent class. In \texttt{JavaScript}, there are no classes, only dynamic objects, so \texttt{JavaScript} uses another form of inheritance. Each \texttt{JavaScript} object has a \emph{prototype object}. The prototype describes the methods and properties assigned to the object. It is used by the dynamic type system to determine whether a particular object contains a method or a property. This prototype can in turn be linked to a parent prototype, forming a prototype inheritance chain. To create an object and link it to another objects prototype, i.e. create an inheritance relation, \texttt{JavaScript} provides the \texttt{Object.create()} method (listing~\ref{inheritance}).

\begin{lstlisting}[caption=Inheritance,label=inheritance]
var a = {field:'hello'};
var b = Object.create(a);  // b inherits from a
console.log(b.field);      // prints 'hello'
a.field = 'hi';
console.log(b.field);      // prints 'hi'
b.field = 'bye';           // now b.field will override a.field
console.log(b.field)       // prints 'bye'
\end{lstlisting}	
\paragraph{}
This mechanism is more powerful but less robust than the \texttt{Java} class inheritance. Since the connection between the parent and child is dynamic, any change in the parent object propagates to the child, even after the child has been created (listing~\ref{inheritance}). On the other hand, changes in the child do not propagate to the parent. It may break the link to the parent which can make this type of inheritance more error prone than classical inheritance (\ref{inheritance}). However the prototypal inheritance is more powerful than its classical counterpart. One example is that you can relatively easily emulate classical inheritance using prototypal inheritance but not vice versa\cite{mozilla}.


%%% TODO title ??
\subsection{The function object}
\paragraph{}
One very powerful feature of \texttt{JavaScript} is the function object. In \texttt{JavaScript}, there is no dedicated function type but the function type inherits from \texttt{Object}. Functions can be treated as objects which makes the language considerably more powerful. Functions treated like objects enables the use of higher order functions :  \texttt{JavaScript} is \emph{both} an object oriented and a functional language. A small example of the advanced concepts made possible by this multi paradigm is shown in listing~\ref{functionobject}.

%% TODO : title ???
\begin{lstlisting}[caption=Inheritance,label=functionobject]
function map(f,array){
	for (var i=0;i<array.length; i++){
		array[i] = f(array[i]);
	}
	return array;
}
function add2(x){
	return x+2;
}
map(add2,[1,2,4]); // will return [3,4,6]
\end{lstlisting}	

\section{Security}
%%	- XSS
%%	- Compare with Java
%%	- Sandboxing
%%  - the importance of validation
%%  - the eval() statement
%%  - malformed html inclusion
%%JavaScript can be used to validate data in HTML forms before sending off the content to a server.
%%Form data that typically are checked by a JavaScript could be:
%%    has the user left required fields empty?
%%    has the user entered a valid e-mail address?
%%    has the user entered a valid date?
%%    has the user entered text in a numeric field?
%%
%%

\section{Conclusion}

\end{document}
