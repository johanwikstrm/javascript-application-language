\documentclass{report}

\usepackage{extension}
\pagestyle{fancy}
\fancyhead{}
\fancyhead[L]{Johan Wikström, Anaël Bonneton}
\fancyfoot{}

\title{JavaScript and its suitability as an application development language}
\author{Johan Wikström (jwikstrom) - 645714\\
Anaël Bonneton (abonneton) - 646275 
}
\begin{document}
\maketitle
\tableofcontents

\section{Introduction}
\texttt{JavaScript} was created by Brendan Eich in 1995 for the company Netscape Communications Corporation. Nowdays, it is the most used programming language for client-side programming. It has also become common in sever-side programming. However, \texttt{JavaScript} is still quite unpopular among professional developers. The main reason is that it was initially created to make programming easier than with traditional programming languages such as \texttt{Java}. \texttt{JavaScript} was created for beginner programmers.
\paragraph{}
We will focus on the following question \emph{"Is JavaScript suitable for application programming?"} with different points of view : the application design, the type system, and finally the security. 


\section{Application design}

\paragraph{}
The \texttt{JavaScript} syntax is easy, without a lot of contraints like in \texttt{Java}. For example, semi-colons are not compulsary at the end of each statement.

\subsection{Scope of variables}

\paragraph{}
\texttt{Javascript} allows to the programmer to define \emph{local} and \emph{global} variables (listing~\ref{VarDefinition}).
\begin{lstlisting}[caption={Defining variables}, label={VarDefinition}]
    var i = 10; // a local variable
    j = 20;     // a global variable
\end{lstlisting}

\paragraph{}
In \texttt{JavaScript} local variables are declared with the keyword \texttt{var}. If the programmer forgets to add this keyword before the declaration of a variable it is a global variable. Global variables are accessible from anywhere, even in different files loaded on the same page. In the listing~\ref{variables}, even if the variable \texttt{j} is declared inside the function \texttt{f}, this variable is global : it will be accessible everywhere.  
\begin{lstlisting}[caption={Global and local variables}, label={variables}]
function f(x, y) {
    var i = 10; // a local variable
    j = 20;     // a global variable
}
\end{lstlisting}

\paragraph{}
With \texttt{JavaScript} it is very easy to use a lot of global variables, and to create them by accident (forgetting the keyword \texttt{var}). However, this is not a good programming practice. The scope of the global variables is not limited, they can be read and modified everywhere : it may be a source of unreadability. Code is easier to understand and to maintain when the scope of its variables is limited. Besides, it is very easy to overwrite gobal variables unintentionally and create bugs. Finally, it is very hard to reuse your code in another project if it contains a lot of global variables : the risk of collision is too important.

\paragraph{}
With \texttt{JavaScript} there is no scope like in \texttt{Java} : even if a variable is defined inside a function, its scope is not always only the function. Programmers have to be very careful with the scope of variables in \texttt{JavaScript}.

\subsection{Creating an object}
\paragraph{}
\texttt{JavaScript} offers several ways to create an object with its fields and its methods. The listings~\ref{way1},~\ref{way2} and~\ref{way3} show the different options.
        
\begin{lstlisting}[caption={Creating an object with a constructor}, label={way1}]
function coordinates(x, y) {
    // Definition of the fields
    this.x = x;
    this.y = y;
    // Definition of the methods
    this.setX = setX;
    function setX(x) {
        this.x = x;
    }
    this.setY = setY;
    function setY(y) {
        this.y = y;
    }
}
var point = new coordinates(1, 2);
\end{lstlisting}
        
\begin{lstlisting}[caption=Creating an object with the "C-structure" syntax, label={way2}]
// We suppose the functions setX and setY are defined before
var point = {x:1, y:2, setX:setX, setY:setY};
\end{lstlisting}

\begin{lstlisting}[caption=Creating an object with the operator "new Object()", label={way3}]
var point = new Object();
point.x = 1;
point.y = 2;
// We suppose the functions setX and setY are defined before
point.setX = setX;
point.setY = setY;
\end{lstlisting}

Whatever the method used to create the object, it is always possible to add another field or another method dynamically (listing~\ref{addField}).
\begin{lstlisting}[caption=Adding a new field or a new method dynamically", label={addField}]
point.z = 3;
function setZ(z) {
        this.z = z;
    }
point.setZ = setZ;
\end{lstlisting}

\paragraph{}
With the "C-structure" syntax and the operator \texttt{new Object()} it is very easy to develop a \texttt{JavaScript} application without thinking about the design of the application. Non-experienced developers may write code very quickly without thinking about readability for future developers and maintainability. However, the \texttt{JavaScript} constructors can be used as the classes in \texttt{Java} to define all the fields and the methods of an object.


%%	Global object
%%	code structure/design
%%    Syntax and semantics
%%        -blocks
%%            function, 
%%
%%var carname="Volvo";
%%var carname;
%%    -> no error, the variable carname has still the value "Volvo"
%%
%%
%%	- scope
%%The lifetime JavaScript variables starts when they are declared.
%%Local variables are deleted when the function is completed.
%%Global variables are deleted when you close the page.
%%
%%
%%	- semicolon
%%Ending statements with semicolon is optional in JavaScript.
%%	- this
%%	- object vs function
%%	- multi paradigm
%%


\section{Type System}
JavaScript has a loose dynamic type system. This means that when you create a variable, you do not need to specify a type and the type of the variable can change during execution as shown in listings \ref{javastypes} and \ref{javascripttypes} . Internally there is a type however and there are six built-in types:
	%http://www.2ality.com/2013/09/types.html
	\begin{enumerate}
	\item Undefined
	\item Null
	\item Boolean
	\item String
	\item Number
	\item Object
	\end{enumerate}
\begin{lstlisting}[caption=Changing type of variable in Java,label={javatypes}]
int x = 1;
x = "hi";   // invalid, causes compiler error
\end{lstlisting}

\begin{lstlisting}[caption=Changing type of variable in JavaScript,label={javascripttypes}]
var x = 1;  // now x has Number type
x = "hi";   // now x has String type
x = {};     // now x has Object type
\end{lstlisting}
	If a variable's type does not fit the operation that is applied on it, JavaScript will attempt to \emph{coerce} the variable into that type. This can be a source of great confusion and many bugs since it effectively masks errors. But it can make programming easier for beginners since they do not have to worry about bitness, converting input strings into numbers and some sorts of zero inputs.
\begin{lstlisting}[caption=Automatic type coercion]
2 + '10'   == '210' // Number coerced into String
2 * '10'   == 20    // String coerced into Number
true - 10  == -9    // Boolean coerced into Number
\end{lstlisting}

	In general, JavaScript tries to recover from errors made by the programmer, by avoiding throwing exceptions when encountering type errors. For example, when an uninitialized variable is used, there is no error but the variable is assigned the type Undefined and value undefined. This usually causes errors later in the program when the value is used. The errors are amplified by the fact that several normal operations are defined for the undefined value which further delays detection of the error.
\begin{lstlisting}[caption=Normal operations on undefined value]
1+undefined       == NaN
''+undefined      == 'undefined'
undefined && true == undefined
undefined || true == true
\end{lstlisting}	
This loose typing may be a disadvantage of the language when comparing the language to strongly typed languages such as Java. In Java the type system allows for static analysis of the code and many bugs are found at compile time. The true advantage of the loose type system only becomes apparent when the programmer uses the ability to augment types with new methods and properties at runtime.

\subsection{JavaScript Object and prototyping}
In Java, classes are static. There is no way to add a new method or property to an object after it is created. In JavaScript, there are no classes so all methods and properties are dynamically. This is a very powerful feature that may also be the cause of many bugs since properties and methods can be added and removed at any time during the programs execution.

\begin{lstlisting}[caption=JavaScript]
a = {} 	// empty object
a.property = "a"
a.method = function(){print("hello")}
\end{lstlisting}	

\begin{verbatim}

\end{verbatim}

\begin{lstlisting}[caption=Java]
class A {
	public String property;
	public void method(){
		print("hello");
	}
}
A a = new A();
a.property = "a";
a.property2 = "b"; // Invalid, was not specified in class
\end{lstlisting}	
%https://developer.mozilla.org/en-US/docs/Web/JavaScript/Guide/Inheritance_and_the_prototype_chain
Another difference between Java and JavaScript is the way inheritance is implemented. In Java, a class can statically inherit methods and variables from a parent class. In JavaScript, there are no classes, only dynamic objects, so JavaScript uses another form of inheritance. All objects in a JavaScript execution are linked to a prototype object. The prototype describes the methods and properties assigned to the object and this is used by the dynamic type system to determine whether a particular object contains the method or property. This prototype can in turn be linked to a parent prototype, forming a prototype inheritance chain. To create an object and link it to another objects prototype,i.e. create an inheritance relation, JavaScript provides the Object.create() method as shown in listing~\ref{inheritance}.

\begin{lstlisting}[caption=Inheritance,label=inheritance]
var a = {prop:'hello'};
var b = Object.create(a); // b inherits from a
console.log(b.prop);      // prints 'hello'
a.prop = 'hi';
console.log(b.prop);      // prints 'hi'
b.prop = 'bye';           // now b.prop will override a.prop
console.log(b.prop)       // prints 'bye'
\end{lstlisting}	
This mechanism is more powerful but less robust than the Java class inheritance. Since the connection between the parent and child is dynamic, any changes in the parent object propagates to the child even after the child is created, as shown in listing~\ref{inheritance}. But on the other hand, changes in the child does not propagate to the parent and may break the link to the parent as shown in listing~\ref{inheritance} which can make this type of inheritance more error prone than classical inheritance. 

But the prototypal inheritance is more powerful than its classical counterpart. One example of this is that you can relatively easily emulate classical inheritance using prototypal inheritance but not vice versa.



\subsection{The function object}
One very powerful feature of JavaScript is the function object. In JavaScript, there is no dedicated function type but the function type inherits from Object. This means that functions can be treated as objects which makes the language considerably more powerful. Functions treated like objects enables the use of higher order functions, making JavaScript into \emph{both} an object oriented and functional language enabling very advanced concepts. A small example of this power is shown in listing~\ref{functionobject}

\begin{lstlisting}[caption=Inheritance,label=inheritance]
function map(f,array){
	for (var i=0;i<array.length; i++){
		array[i] = f(array[i]);
	}
	return array;
}
function add2(x){
	return x+2;
}
map(add2,[1,2,4]); // will return [3,4,6]
\end{lstlisting}	

\section{Security}
When you load a webpage on your browser code may be executed on your computer. Security is very important to prevent this code from damaging your data or accessing private information.

\subsection{Restricted features}

\paragraph{}
\texttt{JavaScript} does not provide any function to delete or modify a file, or a directory on the client computer. Then, a \texttt{JavaScript} program cannot delete one of the user's file, or plant a virus in the user's system.

\paragraph{}
Besides, \texttt{JavaScript} does not provide any networking primitive : it cannot estalish a connection to another host. Thanks to this restriction, a \texttt{JavaScript} program cannot use a user's computer in order to crack passwords on another machine.

\paragraph{}
However, even if \texttt{JavaScript} programming language does not provide filesystem and networking functions, a \texttt{JavaScript} program may execute such functions via \texttt{ActiveX} for example.

\subsection{The Same Origin Policy (SOP)}

\paragraph{}
    A JavaScript is only allowed to read and/or write access to properties of elements, windows, or documents that share the same origin with the script. The origin of an element is defined by the \emph{protocol}, \emph{domain} and the \emph{port} that were used to access this element. See table~\ref{SOP} for examples. %% todo REFERENCE BIBLIO
\begin{table}[h!]
\begin{tabular}{|l|l|}
      \hline
      URL & Outcome\\
      \hline\hline
    http://store.company.com/dir2/other.html        & Success \\
    http://store.company.com/dir/inner/another.html & Success \\
    https://store.company.com/secure.html           & Failure (different protocol) \\
    http://store.company.com:81/dir/etc.html        & Failure (different port) \\
    http://news.company.com/dir/other.html          & Failure (different host) \\      
      \hline
\end{tabular}
\caption{The SOP for the URL http://store.company.com/dir/page.html}\label{SOP}
\end{table}

\paragraph{}
The \emph{Same Origin Policy} is a really important security restriction in \texttt{JavaScript}. It allows for example to ensure the user privacy. %% TODO other examples

\paragraph{}
This rule is a severe restriction and ensures the user privacy. Thanks to the Same Origin Policy a malicious script in one window cannot use DOM methods to access the content of documents in other browser windows. %% todo MORE explanations

\paragraph{}
However, this rule is sometimes too restrictive. Large web sites may use several servers. For example, a script from \texttt{home.netscape.com} might want to access the content of a document loaded from \texttt{developer.netscape.com}. \texttt{JavaScript 1.1} answers this problem thanks to the \texttt{domain} property of the Document object. By default, the \texttt{domain} property is the hostname of the server from which the document was loaded. However, the developer can set this property to a string that is a valid domain suffix of itself. For the example, the default value of the \texttt{domain} property is \texttt{home.netscape.com}, and in order to be able to access the content of a document loaded from \texttt{developer.netscape.com} the developer can set this property to \texttt{netscape.com}.
\subsection{Cross site scripting}
Cross site scripting is one of the most common security breaches today. It is a generic term describing the insertion of malicious code into a webpage. Cross site scripting is not limited to JavaScript but the language has several features that facilitate cross site scripting.

One way to insert malicious code into a web page is to trick the server into inserting the correct HTML tags into a dynamic web page. This is facilitated by the fact that JavaScript can be inserted and run anywhere on a web page by using the \texttt{<script>} tag. One example of this can be shown in listing~\ref{htmlinsertion} and the problem is amplified by browser makers that are trying to make sense out of broken script tags and run them. So even if the web server has some sort of protection of script inclusion, it is often insufficient.
\begin{lstlisting}[caption={Insertion of malicious script on the server side},label={htmlinsertion},language={html}]
<!-- the user has previously submitted his name to the application -->
<html>
<body>
<!-- Here Mr Anderson is fetched from a database-->
<p>Hello Mr Anderson</p> 
</body>
</html>
<!--malicious case, the user has entered code instead of a name -->
<html>
<body>
<p>Hello <script> evilCode();</script></p>
</body>
</html>
\end{lstlisting}

Another form of cross site scripting is the result of one unfortunate language feature. JavaScript has the ability to evaluate and run JavaScript using the eval() function. As demonstrated in listing~\ref{eval}, the eval function takes a string and runs the JavaScript code contained in it. In combination with the many global variables in JavaScript, this is disastrous from a security perspective. The use of eval() is strongly discouraged but it is still used by beginners, mainly for parsing JSON data.
\begin{lstlisting}[caption={Insertion of malicious script on the client side, eval().},label={eval}]
eval('var a = 5');              // creates variable a
var jsonString = getJSON();
eval(jsonString);                // evaluates possibly harmful input
JSON.parse(jsonString);          // safe way to do the same thing
\end{lstlisting}


\paragraph{}
%% Conclusion for security
if you write a webapplication in JavaScript, you cannot be careful about all the security issues.
the browser -> CSRF
the user -> ???

%%	- XSS cross site scripting - the eval() statement, java (Johan)
%%	- Security model, Same origin policy, java
%%  - malformed html inclusion, java


%%	- Security model, Same origin policy, java
%%  
%%  - global variables, (Anael)
%%

\section{Conclusion}
JavaScript has several features that makes the language unsuitable for application development. It has a type system which masks errors, a multitude of security issues and it enables beginner programmers to start programming without knowing the language. There are some features that should never have been allowed into the language, such as the focus on global variables and the eval() function.

But JavaScript also contains some very powerful features that, if used right, makes the language suitable for application development. These features include inheritance, closure and high level functions. All of these advanced features are not even present in JavaScript's big brother Java.

To create a well designed application in JavaScript is hard, because the language allows you to get away with poor designs. But if you use the powerful features of the language (and avoid the bad ones) it allows you to create just as good, if not better, programs as any other high level language.
\begin{thebibliography}{99}

\end{thebibliography}

\end{document}
