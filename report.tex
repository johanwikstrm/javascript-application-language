\documentclass{report}

\usepackage{extension}
\pagestyle{fancy}
\fancyhead{}
\fancyhead[L]{Johan Wikström, Anaël Bonneton}
\fancyfoot{}

\title{JavaScript and its suitability as an application development language}
\author{Johan Wikström (jwikstrom) - 645714\\
Anaël Bonneton (abonneton) - 646275 
}
\begin{document}
\maketitle
\tableofcontents

\section{Introduction}
\paragraph{}
\texttt{JavaScript} was created by Brendan Eich in 1995 for the company Netscape Communications Corporation. Today, it is the most used programming language for client-side programming on the web\cite{w3techs}. It has also become common in server-side programming. However, \texttt{JavaScript} has a bad reputation among professional developers. The main reason is that it was initially created to make programming easier than with traditional programming languages such as \texttt{Java}. \texttt{JavaScript} was created for beginner programmers and had several features that were not appealing to professional developers.
\paragraph{}
In this report we will answer the following question \emph{"Is JavaScript suitable for application programming?"}. We will focus on three areas: application design, the type system and the security issues.


%%
%%
%%The DOM (Document Object Model) is the official W3C standard for accessing HTML elements. 
%%
\section{Application design}

\paragraph{}
The \texttt{JavaScript} syntax is easy, without a lot of contraints like in \texttt{Java}. For example, semi-colons are not compulsary at the end of each statement.

\subsection{Semi-colons}

\subsection{Global and local variables}

\subsection{Creating an object}
\paragraph{}
\texttt{JavaScript} offers several ways to create an object with its fields and its methods. The listings~\ref{way1},~\ref{way2} and~\ref{way3} show the different options.
        
\begin{lstlisting}[caption={Creating an object with a constructor}, label={way1}]
function coordinates(x, y) {
    // Definition of the fields
    this.x = x;
    this.y = y;
    // Definition of the methods
    this.setX = setX;
    function setX(x) {
        this.x = x;
    }
    this.setY = setY;
    function setY(y) {
        this.y = y;
    }
}
var point = new coordinates(1, 2);
\end{lstlisting}
        
\begin{lstlisting}[caption=Creating an object with the "C-structure" syntax, label={way2}]
// We suppose the functions setX and setY are defined before
var point = {x:1, y:2, setX:setX, setY:setY};
\end{lstlisting}

\begin{lstlisting}[caption=Creating an object with the operator "new Object()", label={way3}]
var point = new Object();
point.x = 1;
point.y = 2;
// We suppose the functions setX and setY are defined before
point.setX = setX;
point.setY = setY;
\end{lstlisting}

Whatever the method used to create the object, it is always possible to add another field or another method dynamically (listing~\ref{addField}).
\begin{lstlisting}[caption=Adding a new field or a new method dynamically", label={addField}]
point.z = 3;
function setZ(z) {
        this.z = z;
    }
point.setZ = setZ;
\end{lstlisting}

\paragraph{}
With the "C-structure" syntax and the operator \texttt{new Object()} it is very easy to develop a \texttt{JavaScript} application without thinking about the design of the application. Non-experienced developers may write code very quickly without thinking about readability for future developers and maintainability. However, the \texttt{JavaScript} constructors can be used as the classes in \texttt{Java} to define all the fields and the methods of an object.


%%	Global object
%%	code structure/design
%%    Syntax and semantics
%%        -blocks
%%            function, 
%%
%%var carname="Volvo";
%%var carname;
%%    -> no error, the variable carname has still the value "Volvo"
%%
%%
%%	- scope
%%The lifetime JavaScript variables starts when they are declared.
%%Local variables are deleted when the function is completed.
%%Global variables are deleted when you close the page.
%%
%%
%%	- semicolon
%%Ending statements with semicolon is optional in JavaScript.
%%	- this
%%	- object vs function
%%	- multi paradigm
%%



\section{Type System}
	JavaScript has a loose dynamic type system. This means that when you create a variable, you do not need to specify a type and the type of the variable can change during execution. Internally there is a type however and there are six internal types:
	%http://www.2ality.com/2013/09/types.html
	\begin{enumerate}
	\item Undefined
	\item Null
	\item Boolean
	\item String
	\item Number
	\item Object
	\end{enumerate}
		\begin{verbatim}
	Java
	int x = 1;
	x = "hi";   // invalid, causes compiler error
	\end{verbatim}
	\begin{verbatim}
	JavaScript
	var x = 1;  // now x has Number type
	x = "hi";   // now x has String type
	x = {};     // now x has Object type
	\end{verbatim}
	If a variables type does not fit the operation that is applied on it, JavaScript will attempt to \emph{corece} the variable into that type.
	\begin{verbatim}
	'1'+'10' == 11 // true in javascript
	\end{verbatim}
	In general, JavaScript tries to recover from errors made by the programmer, by almost never throwing exceptions when encountering type errors. For example, when an uninitialized variable is used, there is no error but the variable is assigned the type Undefined and value undefined. This usually causes errors much later in the program when the value is used. The errors are exaggerated by the fact that several normal operations are defined for the undefined value which further delays detection of the error.
	\begin{verbatim}
	1+undefined == NaN
	''+undefined == 'undefined'
	undefined && true == undefined
	undefined || true == true
	\end{verbatim}
This loose typing may be a disadvantage of the language when comparing the language to strongly typed languages such as Java. In Java the type system allows for static analysis of the code and many bugs are found at compile time. The true advantage of the loose type system only becomes apparent when the programmer uses the ability to augment types with new methods and properties at runtime.

	\subsection{JavaScript Object and prototyping}
	In Java, classes are static. There is no way to add a new method or property to an object after it is created. In JavaScript, there are no classes so all methods and properties are dynamically. This is a very powerful feature that may also be the cause of many bugs since properties and methods can be added and removed at any time during the programs execution.
\begin{verbatim}
JavaScript
a = {} 	// empty object
a.property = "a"
a.method = function(){print("hello")}
\end{verbatim}

\begin{verbatim}
Java
class A {
	public String property;
	public void method(){
		print("hello");
	}
}
A a = new A();
a.property = "a";
a.property2 = "b"; // Invalid, was not specified in class
\end{verbatim}
%https://developer.mozilla.org/en-US/docs/Web/JavaScript/Guide/Inheritance_and_the_prototype_chain
Another difference between Java and JavaScript is the way inheritance is implemented. In Java, a class can statically inherit methods and variables from a parent class. In JavaScript, there are no classes, only dynamic objects, so JavaScript uses another form of inheritance. All objects in a JavaScript execution are linked to a prototype object. The prototype describes the methods and properties assigned to the object and this is used by the dynamic type system to determine whether 
	One feature
	- inheritance
	- loose typing
	- dynamic


\section{Security}
%%	- XSS
%%	- Compare with Java
%%	- Sandboxing
%%        - the importance of validation
%%JavaScript can be used to validate data in HTML forms before sending off the content to a server.
%%Form data that typically are checked by a JavaScript could be:
%%    has the user left required fields empty?
%%    has the user entered a valid e-mail address?
%%    has the user entered a valid date?
%%    has the user entered text in a numeric field?
%%
%%

\section{Conclusion}

\end{document}
