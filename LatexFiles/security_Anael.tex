When you load a webpage on your browser code may be executed on your computer. Security is very important to prevent this code from damaging your data or accessing private information.

\subsection{Restricted features}

\paragraph{}
\texttt{JavaScript} does not provide any function to delete or modify a file, or a directory on the client computer. Then, a \texttt{JavaScript} program cannot delete one of the user's file, or plant a virus in the user's system.

\paragraph{}
Besides, \texttt{JavaScript} does not provide any networking primitive : it cannot estalish a connection to another host. Thanks to this restriction, a \texttt{JavaScript} program cannot use a user's computer in order to crack passwords on another machine.

\paragraph{}
However, even if \texttt{JavaScript} programming language does not provide filesystem and networking functions, a \texttt{JavaScript} program may execute such functions via \texttt{ActiveX} for example.

\subsection{The Same Origin Policy (SOP)}

\paragraph{}
    A JavaScript is only allowed to read and/or write access to properties of elements, windows, or documents that share the same origin with the script. The origin of an element is defined by the \emph{protocol}, \emph{domain} and the \emph{port} that were used to access this element. See table~\ref{SOP} for examples. %% todo REFERENCE BIBLIO
\begin{table}[h!]
\begin{tabular}{|l|l|}
      \hline
      URL & Outcome\\
      \hline\hline
    http://store.company.com/dir2/other.html        & Success \\
    http://store.company.com/dir/inner/another.html & Success \\
    https://store.company.com/secure.html           & Failure (different protocol) \\
    http://store.company.com:81/dir/etc.html        & Failure (different port) \\
    http://news.company.com/dir/other.html          & Failure (different host) \\      
      \hline
\end{tabular}
\caption{The SOP for the URL http://store.company.com/dir/page.html}\label{SOP}
\end{table}

\paragraph{}
The \emph{Same Origin Policy} is a really important security restriction in \texttt{JavaScript}. It allows for example to ensure the user privacy. %% TODO other examples

\paragraph{}
This rule is a severe restriction and ensures the user privacy. Thanks to the Same Origin Policy a malicious script in one window cannot use DOM methods to access the content of documents in other browser windows. %% todo MORE explanations

\paragraph{}
However, this rule is sometimes too restrictive. Large web sites may use several servers. For example, a script from \texttt{home.netscape.com} might want to access the content of a document loaded from \texttt{developer.netscape.com}. \texttt{JavaScript 1.1} answers this problem thanks to the \texttt{domain} property of the Document object. By default, the \texttt{domain} property is the hostname of the server from which the document was loaded. However, the developer can set this property to a string that is a valid domain suffix of itself. For the example, the default value of the \texttt{domain} property is \texttt{home.netscape.com}, and in order to be able to access the content of a document loaded from \texttt{developer.netscape.com} the developer can set this property to \texttt{netscape.com}.


\paragraph{}
%% Conclusion for security
if you write a webapplication in JavaScript, you cannot be careful about all the security issues.
the browser -> CSRF
the user -> ???
