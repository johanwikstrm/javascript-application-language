When you load a webpage on your browser code may be executed on your computer. Security is very important to prevent this code from damaging your data or accessing private information.

\subsection{Restricted features}

\paragraph{}
\texttt{JavaScript} does not provide any function to delete or modify a file, or a directory on the client computer. Then, a \texttt{JavaScript} program cannot delete one of the user's file, or plant a virus in the user's system.

\paragraph{}
Besides, \texttt{JavaScript} does not provide any networking primitive : it cannot estalish a connection to another host. Thanks to this restriction, a \texttt{JavaScript} program cannot use a user's computer in order to crack passwords on another machine.

\paragraph{}
However, even if \texttt{JavaScript} programming language does not provide filesystem and networking functions, a \texttt{JavaScript} program may execute such functions via \texttt{ActiveX} for example.

\subsection{The Same-Origin Policy}
The \emph{Same-Origin Policy} is a really important security restriction in \texttt{JavaScript} :
\begin{center} 
A script can read only the properties of windows and documents that have the same origin (i.e., that were loaded from the \emph{same host}, through the \emph{same port}, and by the \emph{same protocol}) as the script itself. 
\end{center} 


\subsection{User privacy}
\texttt{JavaScript} applications must not be allowed to export private information about the user. 
