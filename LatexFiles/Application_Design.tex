
\paragraph{}
The \texttt{JavaScript} syntax is easy, without a lot of contraints like in \texttt{Java}. For example, semi-colons are not compulsary at the end of each statement.

\subsection{Semi-colons}

\subsection{Global and local variables}

\subsection{Creating an object}
\paragraph{}
\texttt{JavaScript} offers several ways to create an object with its fields and its methods. The listings~\ref{way1},~\ref{way2} and~\ref{way3} show the different options.
        
\begin{lstlisting}[caption={Creating an object with a constructor}, label={way1}]
function coordinates(x, y) {
    // Definition of the fields
    this.x = x;
    this.y = y;
    // Definition of the methods
    this.setX = setX;
    function setX(x) {
        this.x = x;
    }
    this.setY = setY;
    function setY(y) {
        this.y = y;
    }
}
var point = new coordinates(1, 2);
\end{lstlisting}
        
\begin{lstlisting}[caption=Creating an object with the "C-structure" syntax, label={way2}]
// We suppose the functions setX and setY are defined before
var point = {x:1, y:2, setX:setX, setY:setY};
\end{lstlisting}

\begin{lstlisting}[caption=Creating an object with the operator "new Object()", label={way3}]
var point = new Object();
point.x = 1;
point.y = 2;
// We suppose the functions setX and setY are defined before
point.setX = setX;
point.setY = setY;
\end{lstlisting}

Whatever the method used to create the object, it is always possible to add another field or another method dynamically (listing~\ref{addField}).
\begin{lstlisting}[caption=Adding a new field or a new method dynamically", label={addField}]
point.z = 3;
function setZ(z) {
        this.z = z;
    }
point.setZ = setZ;
\end{lstlisting}

\paragraph{}
With the "C-structure" syntax and the operator \texttt{new Object()} it is very easy to develop a \texttt{JavaScript} application without thinking about the design of the application. Non-experienced developers may write code very quickly without thinking about readability for future developers and maintainability. However, the \texttt{JavaScript} constructors can be used as the classes in \texttt{Java} to define all the fields and the methods of an object.


%%	Global object
%%	code structure/design
%%    Syntax and semantics
%%        -blocks
%%            function, 
%%
%%var carname="Volvo";
%%var carname;
%%    -> no error, the variable carname has still the value "Volvo"
%%
%%
%%	- scope
%%The lifetime JavaScript variables starts when they are declared.
%%Local variables are deleted when the function is completed.
%%Global variables are deleted when you close the page.
%%
%%
%%	- semicolon
%%Ending statements with semicolon is optional in JavaScript.
%%	- this
%%	- object vs function
%%	- multi paradigm
%%
